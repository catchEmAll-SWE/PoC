\section{Glossario Piano di Progetto}

\paragraph{PoC}~\smallskip \\
Realizzazione incompleta o abbozzata di un determinato prodotto, allo scopo di provarne la fattibilità o dimostrare la fondatezza di alcuni principi o concetti costituenti.

\paragraph{UML}~\smallskip \\
Acronimo di Unified Modeling Language, è stato creato per realizzare un linguaggio di modellazione visivo comune, ricco 
sia nella semantica che nella sintassi, per l'architettura, la progettazione e l'implementazione di sistemi 
software complessi sia dal punto di vista strutturale che comportamentale. (internet)

\paragraph{Diagramma di Gantt}~\smallskip \\
Strumento molto utile per rappresentare e visualizzare graficamente le tempistiche e l'avanzamento di un progetto.

\paragraph{Technology Baseline}~\smallskip \\
Dimostrare al committente di disporre di librerie, tecnologie e framework utili e neccessari per lo sviluppo di un prodotto.

\paragraph{Milestone}~\smallskip \\
Data del calendario di progetto che indica traguardi importanti durante lo sviluppo di un prodotto.

\paragraph{Design pattern}~\smallskip \\
Può essere definito come "una soluzione progettuale generale ad un problema ricorrente".
Si tratta di una descrizione o modello logico da applicare per la risoluzione di un problema che può 
presentarsi in diverse situazioni durante le fasi di progettazione e sviluppo del software, ancor prima della definizione 
dell'algoritmo risolutivo della parte computazionale. È un approccio spesso efficace nel contenere o ridurre il debito tecnico.

\paragraph{Sprint review}~\smallskip \\
Si tratta della revisione delle attività svolte nello sprint trascorso. Durante la revisione viene controllata la correttezza
dei lavoro scovando le incompletezze e gli errori se presenti.

\paragraph{MVP}~\smallskip \\
Il termine si riferisce al “minimum viable product” ossia alla versione iniziale di un prodotto con 
caratteristiche sufficienti da poter essere utilizzato dai primi clienti, e permette, attraverso i feedback raccolti, di raggiungere lo sviluppo del prodotto finale.

\paragraph{Modello AGILE}~\smallskip \\
La "metodologia agile" indica un insieme di metodi di sviluppo del software direttamente o indirettamente derivati dai principi del "Manifesto per lo sviluppo agile del software".
Tali metodi di sviluppo del software si basano sulla distribuzione continua di software efficienti creati in modo rapido e iterativo, quindi
consente di adottare un approccio più leggero alla stesura della documentazione software e di integrare le modifiche in qualsiasi fase del ciclo di vita, anziché ostacolarle.

\paragraph{Sprint}~\smallskip \\
Intervallo di tempo fisso ripetibile durante il quale viene creato un prodotto "Fatto" del valore più alto possibile.

\paragraph{Product Backlog Refinement}~\smallskip \\
Indica il processo di stima del tempo necessario per i task nel backlog esistente, utilizzando gli story points, 
raffinando i criteri di accettazione per le storie, e dividendo storie più grandi in storie di minore grandezza e complessità.
