\section{Pianificazione}
In questa sezione verrà riportata la pianificazione di progetto prevista dal gruppo Catch Em
All. La pianificazione è stata suddivisa nelle seguenti fasi:
\begin{itemize}
	\item Analisi
	\item Produzione del Proof of Concept
	\item Progettazione architetturale
    \item Progettazione di dettaglio e Codifica
	\item Validazione e Collaudo
\end{itemize}

\subsection{Analisi}
Questa fase ha lo scopo di analizzare in dettaglio il capitolato scelto dal gruppo in modo da
definire gli obiettivi funzionali, i tempi e i costi del progetto, e gli obiettivi di qualità.

\subsubsection{Periodo}
La fase di analisi si svolgerà dal 07/11/2022 fino al 30/01/2023.

\subsubsection{Precondizioni}
\begin{itemize}
	\item E’ stato formato il gruppo n.3: Catch Em All
	\item E’ stato assegnato il capitolato d’appalto C1: “Captcha: umano o sovrumano?”
\end{itemize}

\subsubsection{Postcondizioni}
Stesura e verifica dei seguenti documenti:
\begin{itemize}
	\item Norme di Progetto
	\item Analisi dei Requisiti
	\item Glossario
    \item Piano di Progetto
	\item Piano di Qualifica
\end{itemize}

\subsubsection{Attività}
\begin{itemize}
	\item \textbf{Scelta degli strumenti}: individuazione e studio di tutti gli strumenti utili alla stesura della documentazione e sviluppo del prodotto
    \item \textbf{Norme di Progetto}: stesura del documento contenente le linee guida a cui il gruppo si atterrà per tutte le altre attività di progetto
    \item \textbf{Analisi dei Requisiti}: attività finalizzata alla comprensione dei bisogni espressi nel capitolato d’appalto e ricavati dallo studio del dominio d’uso; i requisiti individuati verranno classificati nel documento Analisi dei Requisiti, il quale conterrà anche i casi d’uso corredati dai diagrammi UML
    \item \textbf{Glossario}: al fine di evitare le ambiguità che si possono creare utilizzando il linguaggio naturale nella stesura dei documenti, le parole chiave utili alla comprensione del dominio d’uso verranno raccolte nel Glossario
    \item \textbf{Piano di Progetto}: stesura del documento che riporta la pianificazione di progetto prevista dal gruppo, la distribuzione delle ore di lavoro e il prospetto dei costi
    \item \textbf{Piano di Qualifica}: stesura del documento dove vengono definiti gli obiettivi di qualità per i processi e i prodotti di progetto, e con quali metodi e strumenti si svolgeranno le attività di verifica e validazione
\end{itemize}

\subsubsection{Ruoli attivi}
Durante la fase di analisi saranno necessari i seguenti ruoli:
\begin{itemize}
	\item Responsabile
    \item Amministratore
    \item Analista
    \item Verificatore
\end{itemize}

\subsubsection{Suddivisione temporale}
La fase di analisi è stata suddivisa in tre periodi distinti, analizzati di seguito.

\subsubsubsection{Primo periodo}
\begin{itemize}
    \item dal 07/11/2022 al 27/11/2022
\end{itemize}
Nel primo periodo il gruppo effettua un’analisi preliminare e avvia le attività di stesura delle bozze dei documenti elencati al paragrafo 4.1.2 Postcondizioni, impostandone la struttura principale. Vengono inoltre scelti gli strumenti da utilizzare per la stesura di tali documenti e redatti i primi verbali in modo da tenere traccia delle riunioni interne e col proponente.

\subsubsubsection{Secondo periodo}
\begin{itemize}
    \item dal 28/11/2022 al 25/12/2022
\end{itemize}
Nel secondo periodo vengono redatti i documenti abbozzati nel primo periodo, partendo dalle Norme di Progetto e Analisi dei Requisiti. A ciascun membro del gruppo vengono affidati dei compiti specifici per ogni sprint. Iniziano anche le attività di verifica incrementale per i documenti in corso di stesura, in modo da monitorare costantemente gli avanzamenti.

\subsubsubsection{Terzo periodo}
\begin{itemize}
    \item dal 26/12/2022 al 08/01/2023
\end{itemize}
Nel terzo periodo il gruppo effettua le attività di verifica finale sui documenti prodotti nel secondo periodo per assicurarsi che i documenti siano coerenti fra loro e conformi alle linee guida stabilite nelle Norme di Progetto e pronti per la revisione RTB.

\subsubsection{Diagramma di Gantt - Analisi}
\includegraphics[width=\textwidth]{src/img/4_analisi.png}\\

\subsection{Produzione del Proof of Concept}
Gli obiettivi di questa fase sono lo studio delle possibili soluzioni architetturali per il PoC e l’individuazione dell’architettura di base per l’implementazione del prodotto. Segue a ciò l’attività di codifica del PoC.
La fase di produzione del Proof of Concept fase termina con la prima revisione RTB.

\subsubsection{Periodo}
La fase di produzione del Proof of Concept si svolgerà dal 09/01/2023 fino al 29/01/2023.

\subsubsection{Precondizioni}
I seguenti documenti sono stati redatti e verificati:
\begin{itemize}
	\item Norme di Progetto
	\item Analisi dei Requisiti
	\item Glossario
    \item Piano di Progetto
	\item Piano di Qualifica
\end{itemize}

\subsubsection{Postcondizioni}
\begin{itemize}
	\item Produzione del PoC
\end{itemize}

\subsubsection{Attività}
\begin{itemize}
	\item \textbf{Individuazione requisiti per il PoC}: attività di analisi finalizzata all’individuazione dei requisiti che il PoC andrà a soddisfare
    \item \textbf{Norme di Progetto}: stesura del documento contenente le linee guida a cui il gruppo si atterrà per tutte le altre attività di progetto
    \item \textbf{Progettazione Technology Baseline}: individuazione dell’architettura di base per l’implementazione del prodotto
        \subitem \textbf{Approfondimento sulle tecnologie scelte}: i membri del gruppo si dedicano allo studio individuale delle tecnologie selezionate; al termine di questa attività tutti avranno acquisito le competenze necessarie per poter lavorare a rotazione sulla produzione del PoC
    \item \textbf{Sviluppo della Technology Baseline}: attività di codifica e verifica del PoC
    \item \textbf{Preparazione della presentazione per la revisione RTB}: : il gruppo si dedica alla preparazione dell’esposizione degli obiettivi raggiunti
\end{itemize}

\subsubsection{Ruoli attivi}
Durante la fase di produzione del Proof of Concept lisi saranno necessari i seguenti ruoli:
\begin{itemize}
	\item Responsabile
    \item Amministratore
    \item Progettista
    \item Programmatore
    \item Verificatore
\end{itemize}

\subsubsection{Suddivisione temporale}
La fase di produzione del Proof of Concept è stata suddivisa in quattro brevi periodi, analizzati di seguito. La milestone individuata è rappresentata dalla revisione RTB.

\subsubsubsection{Primo periodo}
\begin{itemize}
    \item dal 09/01/2023 al 10/01/2023
\end{itemize}
Nel primo periodo vengono individuati i requisiti in base ai quali produrre il PoC e selezionata l’architettura di base per la sua implementazione.

\subsubsubsection{Secondo periodo}
\begin{itemize}
    \item dal 11/01/2023 al 15/01/2023
\end{itemize}
Nel secondo periodo il gruppo si impegna ad approfondire autonomamente le tecnologie scelte nel primo periodo e colmare eventuali lacune nelle conoscenze di strumenti, librerie e così via.

\subsubsubsection{Terzo periodo}
\begin{itemize}
    \item dal 16/01/2023 al 27/01/2023
\end{itemize}
Nel terzo periodo viene effettuata la codifica e verifica del PoC.

\subsubsubsection{Quarto periodo}
\begin{itemize}
    \item dal 28/01/2023 al 30/01/2023
\end{itemize}
Nel quarto periodo il gruppo si dedica alla preparazione della presentazione per la revisione RTB.

\subsubsection{Diagramma di Gantt - Produzione del Proof of Concept}
\includegraphics[width=\textwidth]{src/img/4_produzione.png}\\

\subsection{Progettazione architetturale}
Lo scopo di questa fase è il raffinamento della progettazione architetturale ad alto livello avviata nella fase descritta al paragrafo 4.2 Produzione del Proof of Concept, ovvero “come” saranno soddisfatti i requisiti precedentemente individuati.
Le scelte che il gruppo effettua in questa fase riguarderanno la struttura complessiva del sistema e ne influenzeranno varie caratteristiche qualitative come per esempio l’efficienza, l’estensibilità e la manutenibilità.

\subsubsection{Periodo}
La fase di progettazione architetturale si svolgerà dal 31/01/2023 fino al 05/02/2023.

\subsubsection{Precondizioni}
\begin{itemize}
    \item E’ stato prodotto il PoC
\end{itemize}

\subsubsection{Postcondizioni}
\begin{itemize}
    \item Conclusione della progettazione architetturale ad alto livello
\end{itemize}

\subsubsection{Attività}
\begin{itemize}
    \item \textbf{Incremento e verifica dei documenti}: a seconda delle necessità, il gruppo si occupa di aggiornare la documentazione prodotta in precedenza
    \item \textbf{Progettazione architetturale}: raffinamento della progettazione architetturale ad alto livello
        \subitem \textbf{Approfondimento sulle tecnologie scelte}: i membri del gruppo si dedicano allo studio individuale delle tecnologie selezionate; al termine di questa attività tutti avranno acquisito le competenze necessarie per poter lavorare a rotazione sulla futura realizzazione del prodotto
\end{itemize}

\subsubsection{Ruoli attivi}
Durante la fase di progettazione architetturale saranno necessari i seguenti ruoli:
\begin{itemize}
	\item Responsabile
    \item Amministratore
    \item Progettista
    \item Verificatore
\end{itemize}

\subsubsection{Suddivisione temporale}
La fase di progettazione architetturale è stata suddivisa in due brevi periodi, analizzati di
seguito.

\subsubsubsection{Primo periodo}
\begin{itemize}
	\item dal 31/01/2023 al 01/02/2023
\end{itemize}
Nel primo periodo viene conclusa la progettazione architetturale; le soluzioni scelte punteranno alla correttezza per costruzione.

\subsubsubsection{Secondo periodo}
\begin{itemize}
	\item dal 02/02/2023 al 05/02/2023
\end{itemize}
Nel secondo periodo il gruppo si impegna ad approfondire autonomamente le tecnologie scelte nel primo periodo e colmare eventuali lacune nelle conoscenze di strumenti, librerie e così via.

\subsubsection{Diagramma di Gantt - Progettazione architetturale}
\includegraphics[width=\textwidth]{src/img/4_progettazione.png}\\

\subsection{Progettazione di dettaglio e Codifica}
Questa fase ha lo scopo di avviare le attività riguardanti la progettazione di dettaglio del sistema e la codifica del prodotto.
In particolare, la codifica si svolgerà in base alle norme di codifica stabilite nelle Norme di Progetto e avrà tra gli obiettivi anche l’assicurarsi di scrivere codice facilmente verificabile per facilitare il lavoro della fase successiva, in quanto l'efficacia dei metodi di verifica è strettamente legata alla qualità di strutturazione del codice. In questo modo non sarà necessario dipendere solo dalla verifica retrospettiva, il cui costo cresce con l'avanzare della fase di codifica.

\subsubsection{Periodo}
La fase di progettazione di dettaglio e Codifica si svolgerà dal 06/02/2023 fino al 02/04/2023.

\subsubsection{Precondizioni}
\begin{itemize}
    \item E’ stata conclusa la progettazione architetturale ad alto livello
\end{itemize}

\subsubsection{Postcondizioni}
\begin{itemize}
    \item Conclusione della progettazione di dettaglio
    \item Conclusione della codifica e verifica
\end{itemize}

\subsubsection{Attività}
\begin{itemize}
    \item \textbf{Incremento e verifica dei documenti}: a seconda delle necessità, il gruppo si occupa di aggiornare la documentazione prodotta in precedenza
    \item \textbf{Product baseline}: vengono studiati in dettaglio i design pattern da utilizzare e prodotti relativi diagrammi
        \subitem \textbf{Definizione delle unità software che comporranno il prodotto}: il prodotto viene suddiviso in unità, ciascuna delle quali potrà essere realizzata da un singolo programmatore
    \item \textbf{Codifica}: utilizzando il PoC prodotto in precedenza come base, viene prodotto il restante codice; la codifica avverrà utilizzando un approccio incrementale, per cui ogni incremento sarà costituito dalla codifica di un determinato caso d’uso e produrrà valore aggiunto
        \subitem \textbf{Verifica}: il codice prodotto viene continuamente verificato; quest’attività prepara il successo della fase di validazione
    \item \textbf{Stesura dell’allegato tecnico}: viene prodotto il documento che descrive le caratteristiche architetturali del prodotto
    \item \textbf{Stesura del manuale per la manutenzione del prodotto}: viene prodotto il manuale per la manutenzione e le estensioni future del prodotto
    \item \textbf{Stesura del manuale utente}: viene prodotto il manuale contenente le istruzioni di utilizzo del prodotto
    \item \textbf{Preparazione della presentazione per la revisione PB}: il gruppo si dedica alla preparazione dell’esposizione degli obiettivi raggiunti
\end{itemize}

\subsubsection{Ruoli attivi}
Durante la fase di progettazione di dettaglio e Codifica saranno necessari i seguenti ruoli:
\begin{itemize}
	\item Responsabile
    \item Amministratore
    \item Progettista
    \item Programmatore
    \item Verificatore
\end{itemize}

\subsubsection{Suddivisione temporale}
La fase di progettazione di dettaglio e Codifica è stata suddivisa in tre periodi distinti, analizzati di seguito. La milestone individuata è rappresentata dalla revisione PB.

\subsubsubsection{Primo periodo}
\begin{itemize}
    \item dal 06/02/2023 al 12/02/2023
\end{itemize}
Nel primo periodo viene conclusa la progettazione di dettaglio e iniziata la stesura dell’Allegato tecnico: a questo punto ogni attività di codifica può essere avviata in base alle scelte architetturali fatte dal gruppo.

\subsubsubsection{Secondo periodo}\begin{itemize}
    \item dal 13/02/2023 al 26/03/2023
\end{itemize}
Nel secondo periodo il gruppo si dedica alle attività di codifica e verifica. Ad ogni sprint review vengono analizzati i risultati raggiunti e studiato un piano di azione per lo sprint successivo, in modo da mantenere un’elevata capacità di rispondere alle eventuali problematiche riscontrate. Al termine di questo periodo il MVP è pronto per la revisione PB.

\subsubsubsection{Terzo periodo}
\begin{itemize}
    \item dal 27/03/2023 al 02/04/2023
\end{itemize}
Nel terzo periodo vengono redatti i manuali per la manutenzione e l’utilizzo del prodotto, e viene preparata la presentazione per la revisione PB.

\subsubsection{Diagramma di Gantt - Progettazione di dettaglio e Codifica}
\includegraphics[width=\textwidth]{src/img/4_codifica.png}\\

\subsection{Validazione e Collaudo}
In questa fase vengono creati ed applicati tutti i test necessari per garantire la qualità del prodotto. Il progetto si conclude con una verifica del comportamento del sistema completo rispetto ai requisiti stabiliti in precedenza, in presenza del committente.

\subsubsection{Periodo}
La fase di Validazione e Collaudo si svolgerà dal 03/04/2023 fino al 30/04/2023.

\subsubsection{Precondizioni}
\begin{itemize}
    \item E’ stata conclusa la progettazione di dettaglio
    \item Sono state concluse la codifica e la verifica
\end{itemize}

\subsubsection{Postcondizioni}
\begin{itemize}
    \item Produzione dei test necessari
    \item Esecuzione e superamento di tutti i test
\end{itemize}

\subsubsection{Attività}
\begin{itemize}
    \item \textbf{Incremento e verifica dei documenti}: a seconda delle necessità, il gruppo si occupa di aggiornare la documentazione prodotta in precedenza
    \item \textbf{Validazione e Collaudo}: viene verificato che il prodotto finale soddisfi i requisiti stabiliti tenendo in considerazione anche gli obiettivi di qualità definiti nel Piano di Qualifica
    \item \textbf{Preparazione della presentazione per la revisione CA}: il gruppo si dedica alla preparazione dell’esposizione degli obiettivi raggiunti
\end{itemize}

\subsubsection{Ruoli attivi}
Durante la fase di Validazione e Collaudo saranno necessari i seguenti ruoli:
\begin{itemize}
	\item Responsabile
    \item Amministratore
    \item Programmatore
    \item Verificatore
\end{itemize}

\subsubsection{Suddivisione temporale}
La fase di Validazione e Collaudo è stata suddivisa in due periodi distinti, analizzati di seguito. La milestone individuata è rappresentata dalla revisione CA.

\subsubsubsection{Primo periodo}
\begin{itemize}
    \item dal 03/04/2023 fino al 27/04/2023
\end{itemize}
Nel primo periodo vengono prodotti ed eseguiti tutti i test necessari; è possibile che in questo periodo sia necessario un incremento del codice in base ai risultati dei test.

\subsubsubsection{Secondo periodo}
\begin{itemize}
    \item dal 28/04/2023 fino al 30/04/2023
\end{itemize}
Nel secondo periodo il gruppo si dedica alla preparazione della presentazione per la revisione CA.

\subsubsection{Diagramma di Gantt - Validazione e Collaudo}
\includegraphics[width=\textwidth]{src/img/4_collaudo.png}\\
