\section{Analisi dei rischi}

Grazie ad un attenta analisi dei rischi si pone l'obiettivo di prevedere e mitigare rischi e problematicithe che nascono durante lo sviluppo del progetto, cercando diverse strategie per minimizzarli. 
La gestione dei rischi avviene tramite le 4 attività seguenti:
\begin{itemize}
	\item \textbf{Identificazione} dei possibili eventi che potrebbero causare problemi durante l'avanzamento;
	\item \textbf{Analisi} del problema, tramite una valutazione delle conseguenze negative  che apporta e una stima della manifestazione e della pericolosità;
	\item \textbf{Pianificazione} della metodologia per impedire il verificarsi dei rischi individuati e i comportamenti da adottare nel caso in cui si presentassero;
	\item \textbf{Monitoraggio} per rilevare tutti gli eventuali nuovi rischi che si potrebbero presentare, esso è da effettuare durante tutto il periodo di sviluppo del progetto;
\end{itemize}

I rischi sono stati suddivisi in tre categorie:
\begin{itemize}
	\item Rischi personali;;
	\item Rischi tecnologici;
	\item Rischi organizzativi;
\end{itemize}

\subsection{Rischi personali}

\begin{tabular}{ |p{4cm}|p{10cm}|}
\hline
\multicolumn{2}{|c|}{\textbf{Inesperienza in ambito tecnologico}} \\
\hline
\textbf{Descrizione:}& Nessun componente del team ha un elevata esperienza con le tecnologie scelte per lo sviluppo del progetto.\\
\hline
\textbf{Identificazione:}& Il componente comunica al resto del team i problemi riscontrati.\\
\hline
\textbf{Precauzioni:}& Studio approfondito dei software da utilizzare tramite manuali e tutorial online.\\
\hline
\textbf{Pericolosità:}& Alta.\\
\hline
\textbf{Stima di manifestazione:}& Media.\\
\hline
\textbf{Conseguenze:}& Ritardi o inadempienze nello svolgere i lavori assegnati.\\
\hline
\textbf{Piano di contingenza:}& Il componente riscontrato un problema, dovrà consultare la documentazione ufficiale e/o i tutorial online. In caso di necessità potrà richiedere ai componenti con più esperienza di ragionare insieme ai problemi riscontrati per trovare una soluzione.\\
\hline
\end{tabular}

\vspace{20pt}

\begin{tabular}{ |p{4cm}|p{10cm}|}
\hline
\multicolumn{2}{|c|}{\textbf{Difficoltà nella comunicazione interna}} \\
\hline
\textbf{Descrizione:}& La comunicazione tra i componenti del team non è sempre efficiente causando possibili incomprensioni e difficoltà nella collaborazione. Ogni componente ha degli impegni fissi e straordinari che possono non permettere una comunicazione agevole tramite riunioni vocali.\\
\hline
\textbf{Identificazione:}& Ogni membro del gruppo che deve comunicare un'informazione a una o più persone del gruppo, organizza e propone diverse date per effettuare un meeting straordinario. In caso di impossibilità è tenuto a comunicare con messaggi/documenti nel modo più chiaro possibile tramite una comunicazione asincrona. \\
\hline
\textbf{Precauzioni:}& Il team cerca di suddividere i compiti settimanali in task indipendenti, realizzabili dal singolo individuo senza la dipendenza diretta del resto dei membri del team.\\
\hline
\textbf{Pericolosità:}& Alta.\\
\hline
\textbf{Stima di manifestazione:}& Media.\\
\hline
\textbf{Conseguenze:}& Possibili ritardi nello sviluppo del progetto a causa di comunicazioni scarse.\\
\hline
\textbf{Piano di contingenza:}& Vengono messi a disposizione diversi strumenti per la comunicazione, tra cui l'app messaggistica Whatsapp, la piattaforma Discord e la comunicazione tramite email. Si richiede  a ciascun membro del gruppo di controllare periodicamente questi, per essere costantemente informato e per controllare se vi sono delle comunicazioni per lui.\\
\hline
\end{tabular}

\vspace{20pt}

\begin{tabular}{ |p{4cm}|p{10cm}|}
\hline
\multicolumn{2}{|c|}{\textbf{Difficoltà nella comunicazione esterna}} \\
\hline
\textbf{Descrizione:}& La comunicazione con il proponente che avviene esclusivamente tramite email e meeting non è sempre accessibile a tutti i componenti del team.\\
\hline
\textbf{Identificazione:}& Come già dichiarato ogni componente ha degli impegni fissi e straordinari a cui vanno sovrapposti quelli del proponente per effettuare dei meeting insieme. Inoltre la comunicazione tramite email non è sempre efficiente, poichè non è sempre ottenibile una risposta veloce.\\
\hline
\textbf{Precauzioni:}& Il team tramite comunicazione via email organizza meeting accessibili a tutti i componenti, proponendo diverse date e orari concordate tra il gruppo al proponente, che comunicherà le sue disponibilità.\\
\hline
\textbf{Pericolosità:}& Media.\\
\hline
\textbf{Stima di manifestazione:}& Media.\\
\hline
\textbf{Conseguenze:}& Possibili ritardi nella comunicazione dello stato dello sviluppo del progetto o problemi riscontrati.\\
\hline
\textbf{Piano di contingenza:}& Il team ad ogni necessità di comunicare delle informazioni o domande al proponente, durante il meeting settimanale concorda diverse date e orari accessibili a tutti e tempestivamente le comunica al proponente per organizzare una riunione. In caso di un'impossibilità nella sovvrapposizione degli impegni tra le parti, se possibile si cerca di effettuare una comunicazione efficiente tramite email, altrimenti viene effettuato il meeting nella data proposta dal proponente in cui più componenti del team sono disponibili. A termine della riunione, viene redatto il verbale, come ad ogni meeting, in cui vengono elencati tutti i punti affrontati che sarà poi visualizzato dai componenti mancanti al meeting per renderli informati.\\
\hline
\end{tabular}

\vspace{20pt}

\begin{tabular}{ |p{4cm}|p{10cm}|}
\hline
\multicolumn{2}{|c|}{\textbf{Conflitti interni per lo sviluppo del progetto}} \\
\hline
\textbf{Descrizione:}& Nella tecnologie lasciate libere da utilizzare e nei diversi sistemi da implementare per ottenere il risultato richiesto, è comune avere diversi punti di vista tra i componenti del team.\\
\hline
\textbf{Identificazione:}& Ciascun componente in maniera professionale espone il proprio punto di vista al resto del team.\\
\hline
\textbf{Precauzioni:}& Non è possibile decidere autonomamente una tecnologia da utilizzare o un sistema da implementare senza l'approvazione comune.\\
\hline
\textbf{Pericolosità:}& Alta.\\
\hline
\textbf{Stima di manifestazione:}& Alta.\\
\hline
\textbf{Conseguenze:}& Il capitolato viene svolto in un clima avverso.\\
\hline
\textbf{Piano di contingenza:}& Ogni componente con un diverso punto di vista è invitato a fornire una documentazione al fine motivare la sua idea in maniera professionale. Il team tutto riunito, ascoltate le proposte deciderà in maniera democratica la proposta ritenuta migliore e ciascun componente si adeguerà alla scelta decisa.\\
\hline
\end{tabular}

\subsection{Rischi tecnologici}

\begin{tabular}{ |p{4cm}|p{10cm}|}
\hline
\multicolumn{2}{|c|}{\textbf{Implementazione in diversi motori di ricerca, ognuno con delle proprie caratteristiche.}} \\
\hline
\textbf{Descrizione:}& Ad oggi, per visitare il web è possibile utilizzare diversi motori di ricerca, ognuno diverso tra loro.\\
\hline
\textbf{Identificazione:}& Sviluppato il progetto, bisognerà rilevare errori e bug relativi ad ogni motore di ricerca.\\
\hline
\textbf{Precauzioni:}& Documentazione e studio dei problemi che potrebbero nascere.\\
\hline
\textbf{Pericolosità:}& Media.\\
\hline
\textbf{Stima di manifestazione:}& Media.\\
\hline
\textbf{Conseguenze:}& Bug e errori caratteristici di ogni singolo motore di ricerca.\\
\hline
\textbf{Piano di contingenza:}& Sviluppo consapevole dei problemi esistenti che utilizza soluzioni efficaci e applicabili a tutti i motori di ricerca.\\
\hline
\end{tabular}

\vspace{20pt}

\begin{tabular}{ |p{4cm}|p{10cm}|}
\hline
\multicolumn{2}{|c|}{\textbf{Problemi Hardware}} \\
\hline
\textbf{Descrizione:}& Ciascun membro lavora su un computer in remoto, il quale può essere soggetti a guasti e mancanza di connessione.\\
\hline
\textbf{Identificazione:}& Il membro comunica al resto del team il problema.\\
\hline
\textbf{Precauzioni:}& Tutti i file devono avere una copia di backup in remoto su GitHub.\\
\hline
\textbf{Pericolosità:}& Media.\\
\hline
\textbf{Stima di manifestazione:}& Bassa.\\
\hline
\textbf{Conseguenze:}& Problemi nell'avanzamento del singolo individuo nel progetto.\\
\hline
\textbf{Piano di contingenza:}& Utilizzare un altro dispositivo disponibile oppure nel caso non fosse disponibile rivolgersi all'ateneo per richiedere l'utilizzo di un computer di un laboratorio.\\
\hline
\end{tabular}

\vspace{20pt}

\begin{tabular}{ |p{4cm}|p{10cm}|}
\hline
\multicolumn{2}{|c|}{\textbf{Problemi software}} \\
\hline
\textbf{Descrizione:}& Per svolgere qualsiasi attività inerente al progetto il team utilizza software di terze parti, che sono soggetti a bug e ad non essere operativi e utilizzabili.\\
\hline
\textbf{Identificazione:}& I problemi identificati dovranno essere comunicati al resto del team.\\
\hline
\textbf{Precauzioni:}& Dovranno essere effettuati backup in chiavetta ogni settimana per ridurre al minimo le perdite di dati.\\
\hline
\textbf{Pericolosità:}& Media.\\
\hline
\textbf{Stima di manifestazione:}& Bassa.\\
\hline
\textbf{Conseguenze:}& Perdite di dati e indisponibilità nello svolgere le attività.\\
\hline
\textbf{Piano di contingenza:}& In caso di problematiche gravi e durature, il responsabile della settimana dovrà decidere un software alternativo a quello non più utilizzabile.\\
\hline
\end{tabular}

\subsection{Rischi organizzativi}

\begin{tabular}{ |p{4cm}|p{10cm}|}
\hline
\multicolumn{2}{|c|}{\textbf{Organizzazione dei meeting interni}} \\
\hline
\textbf{Descrizione:}& Ciascun componente ha impegni fissi e straordinari durante tutto l'arco della settimana, non riuscendo ad essere sempre tutti disponibili contemporaneamente per dei meeting interni.\\
\hline
\textbf{Identificazione:}& Ogni membro deve dichiarare gli impegni fissi settimanali al resto del team nei primi meeting effettuati, permettendo una mappatura degli impegni di ciascuna persona in un singolo calendario.\\
\hline
\textbf{Precauzioni:}& La comunicazione di impegni straordinari che non permettono di effettuare il meeting settimanale di tutto il team deve avvenire tramite l'app messaggistica whatsapp.\\
\hline
\textbf{Pericolosità:}& Media.\\
\hline
\textbf{Stima di manifestazione:}& Media.\\
\hline
\textbf{Conseguenze:}& Possibili ritardi e mancanza di comunicazione efficiente del singolo individuo.\\
\hline
\textbf{Piano di contingenza:}& Il team concorda una data e un orario per effettuare un meeting di routine ogni settimana in cui tutti possono partecipare che non si sovrappone con gli impegni fissi di ciascun membro tramite la piattaforma Discord. In caso di impegni straordinari di uno o più membri nell'orario del meeting concordato, il responsabile si occupa di trovare un altro orario accessibile a tutti. In caso di impossibilità, il resto del team effettua il meeting come concordato e si occupa di stilare dettagliatamente in un documento Word tutti i punti affrontati per rendere informata la persona e si richiede un feedback da ritornare alla persona interessata.\\
\hline
\end{tabular}

\vspace{20pt}

\begin{tabular}{ |p{4cm}|p{10cm}|}
\hline
\multicolumn{2}{|c|}{\textbf{Calcolo delle tempistiche e dei relativi costi}} \\
\hline
\textbf{Descrizione:}& A causa dell'inesperienza di ciascun membro del team nello svolgere progetti professionalmente è difficile imporre milestone concrete e raggiungibili nei tempi prefissati. \\
\hline
\textbf{Identificazione:}& Ogni settimana un componente diverso del team si occupa di controllare la situazione delle ore già impiegate rispetto al lavoro svolto e delle ore rimanenti rispetto a l'obiettivo da raggiungere, informando il resto del team della situazione attuale.\\
\hline
\textbf{Precauzioni:}& Ogni membro del gruppo dovrà cercare di portare al termine le task assegnati ad ogni sprint rispettando le scadenze.\\
\hline
\textbf{Pericolosità:}& Alta.\\
\hline
\textbf{Stima di manifestazione:}& Media.\\
\hline
\textbf{Conseguenze:}& Nel caso di sottostima del tempo necessario da impiegare non verrebbe rispettata la scadenza imposta, portando ritardi alla conclusione del progetto e necessità di ulteriori ore a quelle preventivate. Nel caso di una sovrastima si avrebbe fornito al proponente un preventivo comprendente ore non necessarie.\\
\hline
\textbf{Piano di contingenza:}& In caso di sottostima il responsabile avrà il compito di riassegnare le risorse nella maniera più efficace in modo da ridurre al minimo i ritardi e le ore in eccesso per la conclusione del progetto. In caso di sovrastima il gruppo potrà sviluppare tutte le richieste opzionali proposte nel capitolato.\\
\hline
\end{tabular}

\vspace{20pt}

\begin{tabular}{ |p{4cm}|p{10cm}|}
\hline
\multicolumn{2}{|c|}{\textbf{Modifiche in corso d'opera}} \\
\hline
\textbf{Descrizione:}& Durante lo sviluppo del progetto potrebbero nascere delle necessità da parte del team o del proponente di cambiare dei requisiti obbligatori o non.\\
\hline
\textbf{Identificazione:}& Vi è la necessità in questa situazione di avere una comunicazione costante tra il proponente e il team.\\
\hline
\textbf{Precauzioni:}& Il team durante i primi meeting con il proponente, si è posto l'obiettivo di definire in maniera più dettagliata possibile i needs che deve soddisfare il prodotto finale.\\
\hline
\textbf{Pericolosità:}& Alta.\\
\hline
\textbf{Stima di manifestazione:}& Bassa.\\
\hline
\textbf{Conseguenze:}& Nel caso in cui venissero richieste delle modifiche non è garantita la possibilità di rispettare le milestones prefissate da parte del team.\\
\hline
\textbf{Piano di contingenza:}& Il team dovrà comunicare periodicamente lo stadio di sviluppo del progetto e delle task che si appresta a svolgere durante il prossimo periodo.\\
\hline
\end{tabular}


