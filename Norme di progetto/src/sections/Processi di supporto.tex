\section{Processi di supporto}

\subsection{Documentazione}
GitHub dispone di un repository contente documentazione riguardante:
\begin{itemize}
    \item assegnazione appalto (lettera di candidatura)
    \item diario di bordo
    \item ricerche e documentazione prodotta dal team
    \item specifiche tecniche del software
    \item link ai verbali (interni ed esterni)
\end{itemize}
Confluence (strumento JIRA) contiene invece i verbali e i documenti retrospettivi: tale sceltà è stata guidata dalla presenza in questo strumento di template, i quali ne facilitano la scrittura

\subsection{Struttura dei documenti}
Tutti i documenti ufficiali seguono una struttura ben definita cosi da mantenere l'omogeneità. Più precisamente ogni documento è formato da:
\begin{itemize}
    \item \textbf{Fontespizio};
    \item \textbf{Registro delle modifiche};
    \item \textbf{Indice};
    \item \textbf{Contenuto principale}.
\end{itemize}
	
\subsubsection{Fontespizio}
Rappresenta la pagina iniziale del documento ed è strutturato come segue:
\begin{itemize}
    \item \textbf{Logo dell'università}: logo dell'\textit{Università di Padova} posizionato in centro alto della pagina, seguito dalla nomenclatura "Università degli  Studi di Padova";
    \item \textbf{Logo del gruppo}: logo del gruppo, posizionato in centro, subito dopo la nomeclatura dell'università;
    \item \textbf{Nome del gruppo e del progetto}: il nome del gruppo e il nome del progetto in questione, seguito da un recapito email;
    \item \textbf{Nome del documento}: è il titolo del documento, in grassetto e posizionato al centro della pagina;
    \item \textbf{Tabella di descrizione}: è la tabella contenente le informazioni generali del documento.
\end{itemize}

\subsubsection{Registro delle modifiche}
I documenti che sono soggetti alle modifiche continue periodiche sono dotati di un registro che li memorizza. Il registro è fomato cosi:
\begin{itemize}
    \item \textbf{Versione}: indica la versione del documento dopo la modifica;
    \item \textbf{Descrizione}: descrive brevemente la modifica apportata;
    \item \textbf{Data}: indica la data in cui è stata modificata il documento;
\end{itemize}

\subsubsection{Indice}
Per agevolare la lettura, tutti i documenti sono dotati di un indice. Le sezioni sono rappresentati da un numero seguiti dal titolo della sezione, ogni sottosezione deve riportare il numero della sezione madre e poi il numero proprio. I numeri devono partire dall'1.

\subsubsection{Contenuto principale}
La pagina del contenuto è costituita da:
\begin{itemize}
    \item \textbf{Intestazione}: in alto a sinistra deve esserci il nome del gruppo \textit{Catch em All}, in altro a destra si trova il numero e nome della sezione in cui si trova;
    \item \textbf{Pie di pagina}: in basso sinistra si trova il nome del progetto e la sua versione attuale, in basso a destra viene indicato il numero della pagina in cui si trova e il numero di pagine complessive del documento.
\end{itemize}

\subsection{Classificazione dei documenti}
Tutti i documenti prodotti sono divisi in uso interno e uso esterno:
\begin{itemize}
    \item \textbf{Uso interno}: sono documenti usati esclusivamente dal gruppo, tra cui \textit{Norme di progetto} e \textit{Verbali interni};
    \item \textbf{Uso esterno}: sono documenti per i componenti fuori dal gruppo, tra cui \textit{Analisi dei requisiti}, \textit{Verbali esterni}, \textit{Piano di progetto}, (da completare);
\end{itemize}

