\section{Specifiche dei test}

\subsection{Scopo della verifica\textsubscript{G} software}
La verifica\textsubscript{G} software serve per accertare che l'esecuzione delle attività attuate nel periodo in esame non abbia introdotto errori. La forma di verifica software utilizzata dal gruppo \textit{Catch Em All} sarà l'Analisi Dinamica, che viene effettuata tramite test che richiedono l'esecuzione dell'oggetto di verifica. In particolare, i test dovranno essere:
\begin{itemize}
	\item Ripetibili;
	\item Automatizzabili.
\end{itemize}
Gli oggetti della verifica saranno le unità\textsubscript{G} software, le integrazioni tra unità\textsubscript{G}, e anche l'intero sistema. Essendo invece il dominio\textsubscript{G} delle esecuzioni possibili infinito, il gruppo selezionerà un insieme finito di casi da studiare, che dovrà essere sufficiente per garantire la qualità attesa.\\
La verifica\textsubscript{G} software così descritta prepara il successo della validazione\textsubscript{G} software, la quale invece servirà per accertare che il prodotto finale sia conforme alle aspettative.\\
Le specifiche dei test verranno definite nelle prossime versioni del presente documento.
\subsection{Test di unità}
Solitamente un'unità\textsubscript{G} software può essere realizzata da un singolo programmatore, e pertanto il test di unità, che ha il compito di verificare il comportamento di ogni unità\textsubscript{G} isolandola dalle altre, potrà essere a carico dello stesso autore. Il test di unità potrà considerarsi completo una volta che tutte le unità\textsubscript{G} software saranno state verificate.

\subsection{Test di integrazione}
Nei test di integrazione le singole unità\textsubscript{G} software che insieme realizzano una funzionalità del sistema vengono raggruppate nelle componenti individuate nella fase di progettazione architetturale. Servono quindi proprio per rilevare eventuali difetti di progettazione.

\subsection{Test di sistema}
I test di sistema sono finalizzati all'accertamento della copertura dei requisiti\textsubscript{G} individuati nella fase di analisi, e sono quindi test propedeutici al collaudo.

\subsection{Test di regressione}
I test di regressione vengono utilizzati per accertare che le modifiche effettuate per aggiunta, correzione o rimozione, non pregiudichino le funzionalità già verificate in un periodo precedente, causando regressione. Consistono nella ripetizione dei test già definiti ed eseguiti con esito positivo in precedenza.

\subsection{Test di collaudo}
Il test di collaudo saranno supervisionati dal committente, per dimostrazione di conformità del prodotto rispetto alle aspettative.
